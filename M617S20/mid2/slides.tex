\documentclass[10pt,hyperref]{beamer}

% various settings for beamer; note there is a red color theme:
\usetheme{Madrid}
\usecolortheme{beaver}
\setbeamercovered{transparent}
\setbeamerfont{frametitle}{size=\large}
\hypersetup{colorlinks}
\usepackage{times}  % looks better than the default
\setbeamercolor*{block title}{bg=red!10}
\setbeamercolor*{block body}{bg=red!5}

% possibly useful:
\newcommand{\CC}{\mathbb{C}}
\newcommand{\RR}{\mathbb{R}}
\newcommand{\ip}[2]{\left<#1,#2\right>}


% PLEASE JUST CHANGE THE AUTHOR LINE AND LEAVE THE REST OF THE PREAMBLE:
\title{Consequences of the Baire theorem}  % KEEP THIS TITLE
\author{XXX YYY}  % PUT YOUR NAME HERE
\date{\today}  % KEEP THIS


\begin{document}
\beamertemplatenavigationsymbolsempty  % RECOMMENDED TO KEEP

% KEEP THE TITLE SLIDE:
\begin{frame}
  \maketitle
\end{frame}


% SUGGESTED FIRST SLIDE ... RECOMMENDED TO KEEP THIS AS IS
\begin{frame}{the Baire theorem}

\begin{definition}
if $(X,d)$ is a metric space and $S \subset X$ then we say $S$ is \emph{nowhere dense} if the closure $\overline{S}$ contains no (positive radius) balls
\end{definition}

\begin{itemize}
\item equivalently, the interior of the closure ${(\overline{S})}^\circ$ is empty
\end{itemize}

\begin{theorem}[Baire 1899]
a nonempty complete metric space is not a countable union of nowhere dense sets
\end{theorem}

\begin{itemize}
\item that is, if $(X,d)$ is complete and $X=\bigcup_{n=1}^\infty A_n$ then there exists a subset $A_n$ whose closure contains a ball: \quad $B_r(x) \subset \overline{A_n}$ for $x\in X$ and $r>0$
\item the proof of the Baire theorem requires the axiom of choice
\end{itemize}
\end{frame}


% DUPLICATE AND MODIFY THE FOLLOWING SLIDE (FRAME) AS NEEDED
\begin{frame}{TITLE}

CONTENT

\begin{itemize}
\item IT IS OFTEN GOOD TO USE ITEMS FOR SLIDES
\end{itemize}

% block environments you might use inside a frame:
%\begin{definition} CONTENT \end{definition}
%\begin{theorem} CONTENT \end{theorem}
%\begin{lemma} CONTENT \end{lemma}
%\begin{corollary} CONTENT \end{corollary}
\end{frame}

\end{document}
