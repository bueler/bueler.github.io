\documentclass[12pt]{article}
\usepackage[top=1in, bottom=0.7in, left=1.05in, right=1.05in]{geometry}

\usepackage{graphicx,color,enumerate,multicol}
\usepackage{amsmath,amsthm,amsbsy}
\usepackage{palatino}

%% Setup aproblem environment, 
%% aproblem items
%% subproblems environment
%% subproblem items
\makeatletter
\newcounter{probcount}
\newcounter{subprobcount}
\newlength\probsep
\newlength\pshrinking
\newif\iffirstprob
\newenvironment{aproblems}%
  {\ifhmode\unskip\par\fi\setcounter{probcount}{0}\probsep\parskip
  \sbox\@tempboxa{\textbf{9.}}\pshrinking\wd\@tempboxa\advance\pshrinking\labelsep
  \let\hproblem\aproblem
  \advance\linewidth -\pshrinking
  \advance\@totalleftmargin\pshrinking
  \advance\leftskip\pshrinking}%
  {\ifhmode\unskip \par\fi\advance\leftskip-\pshrinking}%

\newcommand{\aproblem}{%
  \setcounter{subprobcount}{0}%
  \stepcounter{probcount}%
  \def\@currentlabel{\arabic{probcount}}%
  \ifhmode
    \unskip \par
  \fi
%  \addpenalty{-4000}%
  \iffirstprob\else\addvspace\probsep\fi
  \firstprobfalse
  \hskip -\labelwidth\hskip -\labelsep 
  \hbox to\labelwidth{\hss\textbf{\arabic{probcount}.}}\hskip\labelsep
}%

\newcommand{\subprob}{\item\def\@currentlabel{\arabic{probcount}\alph{\thelistlabel}}}
\newcommand{\skipproblem}{\stepcounter{probcount}}


%% The following commands put defined left and right headers on the top, and a page number
%% on the bottom of all pages beyond page 1
\usepackage{fancyhdr}
\pagestyle{fancy}
\fancyfoot[C]{\ifnum \value{page} > 1\relax\thepage\fi}
\fancyhead[L]{\ifx\@doclabel\@empty\else\@doclabel\fi}
\fancyhead[R]{\ifx\@docdate\@empty\else\@docdate\fi}
\headheight 15pt
\def\doclabel#1{\gdef\@doclabel{#1}}
\def\docdate#1{\gdef\@docdate{#1}}
\makeatother

%% General formatting parameters
\parindent 0pt
\parskip 6pt plus 1pt


\doclabel{Math F251: Worksheet for Midterm I review}
\docdate{Monday 10 February 2019}


\begin{document}
\renewcommand{\d}{\displaystyle}

\begin{aproblems}
\aproblem (\S 2.6 \#9) \quad  Sketch the graph of a function that satisfies all these conditions:
    $$f(0)=3, \, \lim_{x\to 0^-} f(x) = 4, \, \lim_{x\to 0^+} f(x) = 2, \, \lim_{x\to-\infty} f(x) = -\infty, \hspace{3.0in}$$
    $$\lim_{x\to 4^+} f(x) = \infty, \, \lim_{x\to\infty} f(x) = 3 \hspace{4.0in}$$
\vfill

\aproblem  Find $f'(x)$ using the definition if $f(x)=\sqrt{x}$.
\vfill

\aproblem (\S 2.7 \#7) \quad  Using the result of the last problem, find an equation of the tangent line to $y=\sqrt{x}$ at the point $(1,1)$.
\vfill

\newpage
\aproblem (\S 2.6 \#50) \quad  Find the horizontal and vertical asymptotes of the curve, and state the limits which justify these asymptotes:
    $$y = \frac{1+x^4}{x^2-x^4} \hspace{4.5in}$$
\vfill

\aproblem (\S 2.3 \#49) \quad  Let $g(x) = \frac{x^2+x-6}{|x-2|}$.
\renewcommand{\labelenumi}{(\alph{enumi})}
\begin{enumerate}
\item Find $\lim_{x\to 2^-} g(x)$ and $\lim_{x\to 2^+} g(x)$.
\item Does $\lim_{x\to 2} g(x)$ exist?
\end{enumerate}
\vfill

\aproblem (like \S 2.7 \#53) \quad  The cost of producing $x$ ounces of gold from a new mine is $C=f(x)$ dollars.
\begin{enumerate}
\item What is the meaning of the derivative $f'(x)$?  What are its units?
\item What does the statement $f'(80,000)=17$ mean?
\end{enumerate}
\vfill

\end{aproblems}

\end{document}
