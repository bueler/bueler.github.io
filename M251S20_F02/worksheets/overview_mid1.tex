\documentclass[12pt]{article}
\usepackage[top=1in, bottom=0.7in, left=1.05in, right=1.05in]{geometry}

\usepackage{graphicx,color,enumerate,multicol}
\usepackage{amsmath,amsthm,amsbsy}
\usepackage{palatino}

%% Setup aproblem environment, 
%% aproblem items
%% subproblems environment
%% subproblem items
\makeatletter
\newcounter{probcount}
\newcounter{subprobcount}
\newlength\probsep
\newlength\pshrinking
\newif\iffirstprob
\newenvironment{aproblems}%
  {\ifhmode\unskip\par\fi\setcounter{probcount}{0}\probsep\parskip
  \sbox\@tempboxa{\textbf{9.}}\pshrinking\wd\@tempboxa\advance\pshrinking\labelsep
  \let\hproblem\aproblem
  \advance\linewidth -\pshrinking
  \advance\@totalleftmargin\pshrinking
  \advance\leftskip\pshrinking}%
  {\ifhmode\unskip \par\fi\advance\leftskip-\pshrinking}%

\newcommand{\aproblem}{%
  \setcounter{subprobcount}{0}%
  \stepcounter{probcount}%
  \def\@currentlabel{\arabic{probcount}}%
  \ifhmode
    \unskip \par
  \fi
%  \addpenalty{-4000}%
  \iffirstprob\else\addvspace\probsep\fi
  \firstprobfalse
  \hskip -\labelwidth\hskip -\labelsep 
  \hbox to\labelwidth{\hss\textbf{\arabic{probcount}.}}\hskip\labelsep
}%

\newcommand{\subprob}{\item\def\@currentlabel{\arabic{probcount}\alph{\thelistlabel}}}

\newcommand{\skipproblem}{\stepcounter{probcount}}


%% The following commands put defined left and right headers on the top, and a page number
%% on the bottom of all pages beyond page 1
\usepackage{fancyhdr}
\pagestyle{fancy}
\fancyfoot[C]{\ifnum \value{page} > 1\relax\thepage\fi}
\fancyhead[L]{\ifx\@doclabel\@empty\else\@doclabel\fi}
\fancyhead[R]{\ifx\@docdate\@empty\else\@docdate\fi}
\headheight 15pt
\def\doclabel#1{\gdef\@doclabel{#1}}
\def\docdate#1{\gdef\@docdate{#1}}
\makeatother

%% General formatting parameters
\parindent 0pt
\parskip 6pt plus 1pt


\doclabel{Math F251: Midterm I Overview}
\docdate{Monday 10 February 2020}


\begin{document}
\renewcommand{\d}{\displaystyle}

The midterm will mostly cover Chapter 2, but Chapter 1 skills are needed at all times.  The questions will be like WebAssign and Written Homework problems in Chapter 2.  Note section 2.4 is skipped, along with the ``precise definitions'' material in section 2.6.

\medskip
There are many old versions of Midterm Exam 1 on the ``Exams'' tab on the website.  Print them out and do them, then check your work!

\medskip
Here are important topics which you should review and make sure you understand, with the sections where they appear.  Find and do example problems from each topic!
\begin{itemize}
\setlength\itemsep{1pt}
\item average velocity and secant line slope \hfill \emph{\S 2.1}
\item the sentence definition of the basic (two-sided) limit $\lim_{x\to a} f(x)$ \hfill\emph{\S 2.2}
\item one-sided limits  \hfill \emph{\S 2.2}
\item infinite limits  \hfill \emph{\S 2.2}
\item limits at infinity  \hfill \emph{\S 2.6}
\item vertical and horizontal asymptotes are defined by limits  \hfill \emph{\S 2.2, 2.6}
\item using values close to $x=a$ to estimate the limit  \hfill \emph{\S 2.2}
\item using algebra and limit laws to compute limits  \hfill \emph{\S 2.3, 2.5, 2.6, 2.7, 2.8}
    \begin{itemize}
    \item[$\circ$] you need to do algebra for ``$\frac{0}{0}$,'' ``$\frac{\infty}{\infty}$,'' and ``$\infty - \infty$'' limits
    \item[$\circ$] your basic approach is to cancel zeros, by cancelling, factoring, finding common denominators, or multiplying by conjugates
    \end{itemize}
\item getting limits from a given graph  \hfill \emph{\S 2.2, 2.3, 2.5, 2.6}
\item given limits and values, generate (sketch) a graph  \hfill \emph{\S 2.2, 2.3, 2.5, 2.6}
\item definition of continuity  \hfill \emph{\S 2.5}
    \begin{itemize}
    \item[$\circ$] if a function $f$ is continuous at $x=a$ then the limit $\lim_{x\to a} f(x)$ is easy \dots just plug in $a$
    \end{itemize}
\item common functions are continuous on their domains  \hfill \emph{\S 2.5}
\item using the Intermediate Value Theorem to show equations have solutions \hfill \emph{\S 2.5}
\item the derivative is defined as a limit  \hfill \emph{\S 2.7}
\item computing a derivative from the definition  \hfill \emph{\S 2.7, 2.8}
\item tangent line slope and instantaneous velocity: they are derivatives  \hfill \emph{\S 2.7}
\item find equation of a tangent line  \hfill \emph{\S 2.7}
\item the derivative as a new function derived from $f(x)$  \hfill \emph{\S 2.8}
\item sketching $f'(x)$ based on $f(x)$  \hfill \emph{\S 2.8}
\item notation: $f'(x)=y'=\frac{df}{dx}=\frac{dy}{dx}$  \hfill \emph{\S 2.8}
\item higher derivatives (and notation for them)  \hfill \emph{\S 2.8}
\end{itemize}


\end{document}
