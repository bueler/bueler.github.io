\documentclass{amsart}

\usepackage{verbatim}
\usepackage{color}
\usepackage{palatino}

% inclusion/figure macros
\usepackage{graphicx}

\usepackage{amsmath}

\usepackage[pdftex, colorlinks=true, plainpages=false, linkcolor=blue, citecolor=red, urlcolor=blue]{hyperref}

% macros
\newcommand{\bQ}{\mathbf{Q}}
\newcommand{\bg}{\mathbf{g}}
\newcommand{\br}{\mathbf{r}}
\newcommand{\bu}{\mathbf{u}}
\newcommand{\bU}{\mathbf{U}}
\newcommand{\bx}{\mathbf{x}}
\newcommand{\complex}{\mathbf{C}}
\newcommand{\Div}{\nabla\cdot}
\newcommand{\DivH}{\nabla_H\cdot}
\newcommand{\ddx}[1]{\frac{\partial #1}{\partial x}}
\newcommand{\ddy}[1]{\frac{\partial #1}{\partial y}}
\newcommand{\eps}{\epsilon}
\newcommand{\grad}{\nabla}
\newcommand{\gradH}{\nabla_H}
\newcommand{\ip}[2]{\left<#1,#2\right>}
\newcommand{\lam}{\lambda}
\newcommand{\lap}{\triangle}
%\newcommand{\qed}{\emph{ q.e.d.}}
\newcommand{\real}{\mathbf{R}}
\newcommand{\vf}{\varphi}
%\renewcommand{\note}[1]{\tiny{\textbf{NOTE: #1}.}}

\newcommand{\alert}[1]{{\color{red} #1}}

\begin{document}

\huge
\centerline{IEEE 754: What it means for humanity}

\Large
\centerline{(and us, too)}

\thispagestyle{empty}

\bigskip\bigskip\bigskip\bigskip\bigskip
\Large

The textbook\footnote{L.~Trefethen and D. Bau, \emph{Numerical Linear Algebra}, SIAM Press, 1997.} has an idealized view of floating point, but let me lay out the basics of the real standard.


\begin{itemize}
\setlength\itemsep{1em}
\item See the wikipedia page

 \centerline{\url{en.wikipedia.org/wiki/IEEE_floating_point}}

\item As in the textbook, every representable number is of either form
    $$(-1)^s \times c \times \beta^q$$
for $s\in\{0,1\}$ and integers $\beta,c,q$ where
\begin{gather*}
0 \le c \le \beta^p-1
\end{gather*}
where

\item To convert to the form in the textbook,
    $$\pm \frac{m}{\beta^t} \times \beta^e,$$
FIXME

\item There are several standard formats but the three that matter here are

\begin{tabular}{lllllllll}

\end{tabular}

\begin{comment}
<tr>
<th>Name</th>
<th>Common name</th>
<th>Base</th>
<th>Digits</th>
<th>Decimal<br />
digits</th>
<th>Exponent<br />
bits</th>
<th>Decimal<br />
E max</th>
<th>Exponent<br />
bias<sup id="cite_ref-DAE_4-0" class="reference"><a href="#cite_note-DAE-4"><span>[</span>4<span>]</span></a></sup></th>
<th>E min</th>
<th>E max</th>
</tr>

<tr>
<td><a href="/wiki/Single_precision_floating-point_format" title="Single precision floating-point format" class="mw-redirect">binary32</a></td>
<td>Single precision</td>
<td align="right">2</td>
<td align="right">24</td>
<td align="right">7.22</td>
<td align="right">8</td>
<td align="right">38.23</td>
<td>2<sup>7</sup>−1=127</td>
<td align="right">−126</td>
<td align="right">+127</td>
</tr>

<tr>
<td><a href="/wiki/Double_precision_floating-point_format" title="Double precision floating-point format" class="mw-redirect">binary64</a></td>
<td>Double precision</td>
<td align="right">2</td>
<td align="right">53</td>
<td align="right">15.95</td>
<td align="right">11</td>
<td align="right">307.95</td>
<td>2<sup>10</sup>−1=1023</td>
<td align="right">−1022</td>
<td align="right">+1023</td>
</tr>

<tr>
<td><a href="/wiki/Quadruple_precision_floating-point_format" title="Quadruple precision floating-point format" class="mw-redirect">binary128</a></td>
<td>Quadruple precision</td>
<td align="right">2</td>
<td align="right">113</td>
<td align="right">34.02</td>
<td align="right">15</td>
<td align="right">4931.77</td>
<td>2<sup>14</sup>−1=16383</td>
<td align="right">−16382</td>
<td align="right">+16383</td>
</tr>
\end{comment}

\end{itemize}

\end{document}

