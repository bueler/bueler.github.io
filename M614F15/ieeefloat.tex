\documentclass[11pt]{amsart}
%prepared in AMSLaTeX, under LaTeX2e
\addtolength{\oddsidemargin}{-.75in} 
\addtolength{\evensidemargin}{-.75in}
\addtolength{\topmargin}{-.4in}
\addtolength{\textwidth}{1.6in}
\addtolength{\textheight}{.75in}

\renewcommand{\baselinestretch}{1.05}

\usepackage{verbatim,fancyvrb}

\usepackage{palatino}

\newtheorem*{thm}{Theorem}
\newtheorem*{defn}{Definition}
\newtheorem*{example}{Example}
\newtheorem*{problem}{Problem}
\newtheorem*{remark}{Remark}

\newcommand{\mtt}{\texttt}
\usepackage{alltt,xspace}
\newcommand{\mfile}[1]
{\medskip\begin{quote}\scriptsize \begin{alltt}\input{#1.m}\end{alltt} \normalsize\end{quote}\medskip}

\usepackage[final]{graphicx}
\newcommand{\mfigure}[1]{\includegraphics[height=2.5in,
width=3.5in]{#1.eps}}
\newcommand{\regfigure}[2]{\includegraphics[height=#2in,
keepaspectratio=true]{#1.eps}}
\newcommand{\widefigure}[3]{\includegraphics[height=#2in,
width=#3in]{#1.eps}}

\usepackage{amssymb}

\usepackage[pdftex, colorlinks=true, plainpages=false, linkcolor=blue, citecolor=red, urlcolor=blue]{hyperref}

% macros
\newcommand{\br}{\mathbf{r}}
\newcommand{\bv}{\mathbf{v}}
\newcommand{\bx}{\mathbf{x}}
\newcommand{\by}{\mathbf{y}}

\newcommand{\CC}{\mathbb{C}}
\newcommand{\RR}{\mathbb{R}}
\newcommand{\ZZ}{\mathbb{Z}}

\newcommand{\eps}{\epsilon}
\newcommand{\grad}{\nabla}
\newcommand{\lam}{\lambda}
\newcommand{\lap}{\triangle}

\newcommand{\ip}[2]{\ensuremath{\left<#1,#2\right>}}

%\renewcommand{\det}{\operatorname{det}}
\newcommand{\onull}{\operatorname{null}}
\newcommand{\rank}{\operatorname{rank}}
\newcommand{\range}{\operatorname{range}}

\newcommand{\prob}[1]{\bigskip\noindent\textbf{#1.}\quad }
\newcommand{\exer}[2]{\prob{Exercise #2 in Lecture #1}}

\newcommand{\pts}[1]{(\emph{#1 pts}) }
\newcommand{\epart}[1]{\medskip\noindent\textbf{(#1)}\quad }
\newcommand{\ppart}[1]{\,\textbf{(#1)}\quad }

\newcommand{\Matlab}{\textsc{Matlab}\xspace}

\DefineVerbatimEnvironment{mVerb}{Verbatim}{numbersep=2mm,
frame=lines,framerule=0.1mm,framesep=2mm,xleftmargin=4mm,fontsize=\footnotesize}


\begin{document}
\scriptsize \noindent Math 614 Numerical Linear Algebra (Bueler) \hfill 4 November, 2015
\normalsize

\medskip\bigskip
\Large\centerline{IEEE 754: What it means for humanity}

\large
\centerline{(and us, too)}

\bigskip
\normalsize

\thispagestyle{empty}

\bigskip

The textbook\footnote{L.~Trefethen and D. Bau, \emph{Numerical Linear Algebra}, SIAM Press, 1997.} has an idealized view of floating point, but I think it is a good idea to lay out the basics of the \emph{real} standard.


\begin{itemize}
\setlength\itemsep{1em}
\item See the wikipedia page

 \centerline{\url{en.wikipedia.org/wiki/IEEE_floating_point}}

\item Every representable number is of either form
    $$(-1)^s \times c \times \beta^q$$
for $s\in\{0,1\}$ and integers $\beta,c,q$ where
\begin{gather*}
0 \le c \le \beta^p-1
\end{gather*}
where

\item To convert to the form in the textbook,
    $$\pm \frac{m}{\beta^t} \times \beta^e,$$
FIXME

\item There are several standard formats but the three that matter most all have $\beta=2$.  

\medskip
\small
\begin{tabular}{lllllll}
name & common name & $p$ & exponent bits & exponent bias & $E_{min}$ & $E_{max}$ \\ \hline
%
binary32 &
single &
24 &
8 &
$2^7-1=127$ &
-126 &
+127 \\
%
binary64 &
double &
53 &
11 &
$2^{10}-1=1023$ &
-1022 &
+1023 \\
%
binary128 &
quadruple &
113 &
15 &
$2^{14}-1=16383$ &
-16382 &
+16383
\end{tabular}

\begin{tabular}{llll}
name & decimal digits & decimal $E_{max}$ & decimal $E_{min}$ \\ \hline
%
binary32 &
7.22 &
38.23 &
-37.93 \\
%
binary64 &
15.95 &
307.95 &
-307.65 \\
%
binary128 &
34.02 &
4931.77 &
-4931.47
\end{tabular}

\end{itemize}

\end{document}

