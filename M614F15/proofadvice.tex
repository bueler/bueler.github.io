\documentclass[12pt]{amsart}
%\pagestyle{empty} 
\setlength{\topmargin}{-0.3in} % usually -0.25in
\addtolength{\textheight}{.75in} % usually 1.25in
\addtolength{\oddsidemargin}{-1.0in}
\addtolength{\evensidemargin}{-1.0in}
\addtolength{\textwidth}{2.0in} %\setlength{\parindent}{0pt}

% macros
\usepackage{amssymb,xspace}
\usepackage[final]{graphicx}
\usepackage{hyperref}

\newcommand{\regfigure}[3]{\includegraphics[height=#2in,width=#3in]{#1.eps}}

\newtheorem*{lem*}{Lemma}

\newcommand{\mtt}{\texttt}
\newcommand{\mtl}[1]{{\texttt{>>#1}}}
\usepackage{alltt}
\usepackage{verbatim} % for "comment" environment
\newcommand{\mfile}[1]
{\medskip\begin{quote} \begin{alltt}\input{C:/MATLABR11/work/#1.m}\end{alltt} \end{quote}\medskip}

\newcommand{\CC}{{\mathbb{C}}}
\newcommand{\RR}{{\mathbb{R}}}
\newcommand{\eps}{\epsilon}
\newcommand{\ZZ}{{\mathbb{Z}}}
\newcommand{\ZZn}{{\mathbb{Z}}_n}
\newcommand{\NN}{{\mathbb{N}}}
\newcommand{\bu}{\mathbf{u}}
\newcommand{\bv}{\mathbf{v}}
\newcommand{\ip}[2]{\mathrm{\left<#1,#2\right>}}
\newcommand{\erf}{\operatorname{erf}}
\newcommand{\spn}{\operatorname{span}}

\newcommand{\Matlab}{\textsc{Matlab}\xspace}
\newcommand{\Octave}{\textsc{Octave}\xspace}
\newcommand{\pylab}{\textsc{pylab}\xspace}
\newcommand{\MOP}{\textsc{Matlab}\big|\textsc{Octave}\big|\textsc{pylab}\xspace}

\newcommand{\prob}[1]{\bigskip\bigskip\noindent\large\textbf{#1.} \normalsize}
\newcommand{\bookprob}[1]{\bigskip\bigskip\noindent\large\textbf{Exercise #1.} \normalsize}
\newcommand{\ppart}[1]{\medskip\noindent\large\textbf{\emph{#1})}\normalsize}

\newcommand{\textbook}{\textsc{Trefethen \& Bau}}

\begin{document}
\scriptsize \noindent Math 614 Numerical Linear Algebra (Bueler) \hfill \today
\normalsize

\bigskip\bigskip

\Large\textbf{\centerline{On proving, and on writing proofs}}
\normalsize
\medskip

\thispagestyle{empty}

On assignments you are asked to ``show that \dots'' or ``prove that \dots''.  To do so you should clearly understand the full range of cases you are addressing.  You need to understand what assumptions you may make.  You should understand the conclusion you wish to draw.  (Looking at some particular cases may be they way to get these understandings.)  Then you should make a general, precise, and complete argument which shows that your assumptions imply your conclusion.
\medskip

That is, you need to \emph{prove}.  A proof is a careful argument that reflects complete logical understanding of a situation.
\medskip

For example, suppose an exercise says:
\newcommand{\nprob}[1]{\bigskip\noindent\textbf{#1.}}
\begin{quote}
\nprob{Exercise 666}  Show that if $A$ is an invertible $m\times m$ matrix and if $B$ is an $m\times n$ matrix of full rank, with $n\le m$, then $AB$ has full rank.
\end{quote}
\bigskip

\noindent An appropriate \textbf{solution} starts with a statement (restatement) of what is proved:

\bigskip

\begin{quote}
Suppose $m\ge n$.  Suppose $A\in \CC^{m\times m}$ is an invertible matrix and $B\in \CC^{m\times n}$ is a matrix with full rank.  If $C=AB$ then $C$ has rank $n$.

\begin{proof}  Note that $C\in\CC^{m\times n}$.  Let $v_1,v_2$ be distinct vectors in $\CC^n$.  By Theorem 1.2 in \textbook, because $B$ has full rank, $w_1=B v_1$ and $w_2 = B v_2$ are distinct vectors in $\CC^m$.  By Theorem 1.3, $A$ has full rank, so by Theorem 1.2 $z_1=Aw_1$ and $z_2=Aw_2$ are distinct vectors.  But
	$$z_i = A w_i = A(B v_i) = (AB) v_i = C v_i$$
for $i=1,2$.  Thus $C$ maps distinct vectors $v_1,v_2$ to distinct vectors $z_1,z_2$.  Again by Theorem 1.2, $C$ has full rank.
\end{proof}

\end{quote}\bigskip

\noindent Note these style elements:
\begin{itemize}
\item What I assume is clearly stated.  Do not be afraid to restate the exercise.
\item What I intend to prove (i.e.~the claim ``$C$ has rank $n$'') is clearly stated.
\item The proof is separated from the claim, and its beginning and end are indicated.
\end{itemize}
\medskip

Such a concrete style helps when I am grading your homework.  It helps me determine if your argument does or does not prove the claim, but this style also helps you.  For instance, if you find you cannot prove the most general statement, but you can prove something which (for instance) has stronger assumptions but the same conclusion, then that situation is \emph{clear}.  And you will get an appropriate amount of credit.  I will give much less credit for either a confused statement of what has been proved, or for confused logic inside the proof.
\medskip

I recommend the style of proof used here.  You are not obliged to use it, but you must still make the careful and complete argument.

\end{document}

