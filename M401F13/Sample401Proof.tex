\documentclass{amsart}

% Include any special packages you might use.  Uncomment the
% following to use Times as the default font insteand of
% TeX's default font of Computer Modern.
\usepackage{times,txfonts}

\newcommand{\exer}[1]{\noindent \textbf{Exercise #1}. \,}

% If you want, you can make new commands, e.g. the following
% which uses \Ints to make a blackboard-bold Z.
%
% \newcommand\Ints{\mathbb{Z}}

%%%% Main document starts here.

\begin{document}

\noindent \textbf{Author:} Ed Bueler
\medskip

\noindent \textbf{Date:} 1 September 2013
\bigskip\bigskip

\exer{6.66} The square root of 2 is irrational.
\medskip

\begin{proof}
Suppose to the contrary that $\sqrt{2}$ is rational. Then
there are integers $a$ and $b$ with no common factors such that 
\begin{equation*}
\sqrt{2} = \frac{a}{b}.
\end{equation*}
Squaring this equation we find that
\begin{equation} \label{step1}
2 b^2 = a^2.
\end{equation}
Hence $2$ divides $a^2$ and therefore $a^2$ is even.  
If $a$ were odd, then $a^2$ would also be odd, which it is not, so we conclude
that $a$ is even.  So $a=2k$ for some integer $k$.  It follows from equation
\eqref{step1} that 
\begin{align*}
  2 b^2 &= (2k)^2\\
        &= 4 k^2.
\end{align*}
Hence
\begin{equation*}
b^2 = 2 k^2.
\end{equation*}
Arguing as before we see that $b^2$, and thus also $b$ itself, must be even.
So $2$ is a common factor of $a$ and $b$, which is a contradiction.
\end{proof}

\end{document}
