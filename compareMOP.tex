\documentclass[11pt]{amsart}
\usepackage[margin=1in, head=1in]{geometry}

\usepackage{fancyvrb,xspace}
\usepackage[pdftex,colorlinks=true,urlcolor=blue]{hyperref}

\parindent=0in
\parskip=0.5\baselineskip

%\fvset{fontsize=\small, numbers=left} 
\DefineShortVerb{\|}

\newcommand{\mfile}[1]{
\begin{quote}
\VerbatimInput[frame=single,framesep=3mm,label=\fbox{\normalsize \textsl{\,#1\,}},fontfamily=courier,fontsize=\scriptsize]{#1}
\end{quote}
}

\newcommand{\Matlab}{\textsc{Matlab}\xspace}
\newcommand{\Octave}{\textsc{Octave}\xspace}
\newcommand{\python}{\textsc{Python}\xspace}
\newcommand{\ipython}{\textsc{ipython}\xspace}
\newcommand{\pylab}{\textsc{pylab}\xspace}
\newcommand{\scipy}{\textsc{scipy}\xspace}
\newcommand{\matplotlib}{\textsc{matplotlib}\xspace}

\begin{document}

\title{Comparison of \textsc{Matlab}, \textsc{Octave}, and \textsc{pylab}}

\author{Ed Bueler}

\date{\today.}

\maketitle
\normalsize
\thispagestyle{empty}

\newcommand{\hrf}[2]{\href{#1}{\texttt{#2}}}

On the next page are two algorithms each in \Matlab/\Octave form (left column) and \pylab form (right column).  To download these examples, go to my page \hrf{http://bueler.github.io}{bueler.github.io}.  To get \texttt{gaussint.m} below, for example, go to \hrf{http://bueler.github.io/gaussint.m}{bueler.github.io/gaussint.m}.

A bit of background is useful.  \Matlab (\hrf{http://www.mathworks.com/}{www.mathworks.com}) was designed by Cleve Moler around 1980 for teaching numerical linear algebra without needing FORTRAN.  It has since become a powerful programming language and engineering tool.  More than half of UAF 600-level math/science/engineering students are already familiar with it.  It is available in most labs and graduate student offices at UAF.

But I like free, open source software.  There are several free alternatives to \Matlab, and two of these work well for this course.  First, \Octave is a \Matlab clone.  The ``\texttt{.m}'' examples on the next page work in an identical way in \Matlab and in \Octave.  I will mostly use \Octave myself for teaching, but I'll test examples in both \Octave and \Matlab.  To download \Octave, go to 
\hrf{http://www.gnu.org/software/octave/}{www.gnu.org/software/octave}.

Second, the \scipy (\hrf{http://www.scipy.org/}{www.scipy.org}) and \matplotlib (\hrf{http://matplotlib.org/}{matplotlib.org}) libraries give the general-purpose interpreted language \python (\hrf{http://python.org/}{python.org}) all of \Matlab functionality plus quite a bit more.  This combination is called \pylab.  Using it with the \textsc{ipython} interactive shell (\hrf{http://ipython.org/}{ipython.org}) gives the most \Matlab-like experience.  The examples on the next page hint at the computer language differences and the different modes of thought between \Matlab/\Octave and \python.  Students who already use \python will like this option.

Here are some brief ``how-to'' comments for the \Matlab/\Octave examples: \texttt{gaussint.m} is a \emph{script}.  A script is run by starting \Matlab/\Octave, usually in the directory containing the script you want to run.  Then type the name of the script at the prompt, without the ``.m'':

\verb|>> gaussint|

\noindent Typing

\verb|>> help gaussint|

\noindent shows the block of comments as documentation.

\noindent The second algorithm \verb|bis.m| is a \emph{function} which needs inputs.  At the prompt enter
\begin{Verbatim}
>> f = @(x) cos(x) - x
>> bis(0,1,f)
\end{Verbatim}
for example.  We have given \verb|bis.m| three arguments; the last is an ``anonymous function.''

For the \python versions:  Type \verb|run gaussint.py| at the \ipython prompt or \verb|python gaussint.py| at a shell prompt.  For the function \verb|bis.py|, run \python or \ipython and do: \verb|from bis import bis|.  In \ipython you can type \verb|bis?| to get documentation for that function, and run the example as shown in the docstring.

\vfill


\newpage
\vfill

\begin{tabular}{ll}
\hspace{-0.45in}\parbox{3.6in}{\mfile{gaussint.m}} & \hspace{-0.4in}\parbox{3.8in}{\mfile{gaussint.py}}
\end{tabular}

\begin{tabular}{ll}
\hspace{-0.45in}\parbox{3.6in}{\mfile{bis.m}} & \hspace{-0.4in}\parbox{3.8in}{\mfile{bis.py}}
\end{tabular}

\vfill

\end{document}

