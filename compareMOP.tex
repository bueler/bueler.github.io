\documentclass[11pt]{amsart}
\usepackage[margin=1in, top=0.8in, bottom=0.6in]{geometry}

\usepackage{fancyvrb,xspace}
\usepackage[pdftex,colorlinks=true,urlcolor=blue]{hyperref}

\parindent=0in
\parskip=0.5\baselineskip

%\fvset{fontsize=\small, numbers=left} 
\DefineShortVerb{\|}

\newcommand{\mfile}[1]{
\begin{quote}
\VerbatimInput[frame=single,framesep=3mm,label=\fbox{\normalsize \textsl{\,#1\,}},fontfamily=courier,fontsize=\scriptsize]{#1}
\end{quote}
}

\newcommand{\Matlab}{\textsc{Matlab}\xspace}
\newcommand{\Octave}{\textsc{Octave}\xspace}
\newcommand{\Python}{\textsc{Python}\xspace}
\newcommand{\ipython}{\textsc{ipython}\xspace}
\newcommand{\numpy}{\textsc{numpy}\xspace}
\newcommand{\scipy}{\textsc{scipy}\xspace}
\newcommand{\matplotlib}{\textsc{matplotlib}\xspace}

\begin{document}

\title{Programming languages \\ for the numerical analysis classroom}

\author{Ed Bueler}

\date{\today.}

\maketitle
\normalsize
\thispagestyle{empty}

\newcommand{\hrf}[2]{\href{#1}{\texttt{#2}}}

\Matlab (``matrix laboratory''; see \hrf{http://www.mathworks.com/}{mathworks.com}) was created in the late 1970s by Cleve Moler for teaching numerical linear algebra without requiring FORTRAN programming.  It has since become a powerful programming language and engineering tool.  A large fraction of upper-division and graduate students at UAF are already familiar with it, and it is available through UAF.  The ``Matlab student'' version at \hrf{https://www.mathworks.com/academia/student_version.html}{\texttt{www.mathworks.com/academia/ student\_version.html}}, works fine for the numerical mathematics classes I teach.  It works well, looks good, and I like it.

\Matlab is recommended if you have no existing preference, but I prefer free and open source software.  Among the alternatives are three which work very well for numerical courses:
\renewcommand{\labelenumi}{\arabic{enumi}.}
\begin{enumerate}
\item \Octave is a \Matlab clone.  Download it at
\hrf{http://www.gnu.org/software/octave/}{www.gnu.org/software/octave}.  The ``\texttt{.m}'' examples on the next page, and thoughout this course, work in an identical way in \Matlab and in \Octave.  I will mostly use \Octave myself during the course, but I'll also make sure examples work the same way in \Matlab.
\item The general-purpose language \Python (\hrf{http://python.org/}{python.org}) works very well if you learn to use the \numpy (\hrf{https://numpy.org/}{numpy.org}), \scipy (\hrf{http://www.scipy.org/}{scipy.org}), and \matplotlib (\hrf{http://matplotlib.org/}{matplotlib.org}) libraries.  Using the \textsc{ipython} interactive shell (\hrf{http://ipython.org/}{ipython.org}) gives the most \Matlab-like experience.
\item The Julia language (\hrf{https://julialang.org/}{julialang.org}) is a modern redesign of \Matlab, but it is not a compatible clone like \Octave.  It easy to learn.  Equivalent codes run much faster than in \Matlab or \Octave.
\end{enumerate}

On the next page are two algorithms in \Matlab/\Octave (left column) and \Python (right column); see me for the corresponding Julia examples.  To download these examples, go to

\smallskip

\centerline{\hrf{http://bueler.github.io/M614F21}{bueler.github.io/M614F21}}

\noindent and look in the left column.  Next are some brief ``how-to'' comments.

Program \texttt{gaussint.m} is a \emph{script}.  A script is run by starting \Matlab/\Octave, usually in the directory containing the script you want to run.  Then type the name of the script at the prompt, without the \texttt{.m}:

\verb|>> gaussint|

\noindent Typing \,\verb|help gaussint|\, at the \Matlab/\Octave prompt shows the first block of comments.

\verb|bis.m| is a \emph{function} which needs inputs.  At the prompt enter, for example,
\begin{Verbatim}
>> f = @(x) cos(x) - x
>> bis(0,1,f)
\end{Verbatim}
Note we have given \verb|bis.m| three arguments; the last is a function.

For the \Python codes:  You can do \verb|python gaussint.py| directly from a shell.  Alternatively, from the \Python or \ipython prompt, type \verb|run gaussint|.  For the function \verb|bis.py|, first do: \verb|from bis import bis|.  Then run the example as shown in the docstring; in \ipython you can type \verb|bis?| to print the docstring.

\vfill


\newpage
\vfill

\begin{tabular}{ll}
\hspace{-0.45in}\parbox{3.6in}{\mfile{gaussint.m}} & \hspace{-0.4in}\parbox{3.8in}{\mfile{gaussint.py}}
\end{tabular}

\bigskip
\begin{tabular}{ll}
\hspace{-0.45in}\parbox{3.6in}{\mfile{bis.m}} & \hspace{-0.4in}\parbox{3.8in}{\mfile{bis.py}}
\end{tabular}

\vfill

\end{document}

