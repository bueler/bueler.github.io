\documentclass[12pt]{article}
\usepackage[top=0.9in, bottom=0.5in, left=0.9in, right=1.1in]{geometry}

\usepackage{graphicx,color,enumitem}
\usepackage{amsmath,amsthm,amsbsy}
\usepackage{palatino}

%% Setup aproblem environment, 
%% aproblem items
%% subproblems environment
%% subproblem items
\makeatletter
\newcounter{probcount}
\newcounter{subprobcount}
\newlength\probsep
\newlength\pshrinking
\newif\iffirstprob
\newenvironment{aproblems}%
  {\ifhmode\unskip\par\fi\setcounter{probcount}{0}\probsep\parskip
  \sbox\@tempboxa{\textbf{9.}}\pshrinking\wd\@tempboxa\advance\pshrinking\labelsep
  \let\hproblem\aproblem
  \advance\linewidth -\pshrinking
  \advance\@totalleftmargin\pshrinking
  \advance\leftskip\pshrinking}%
  {\ifhmode\unskip \par\fi\advance\leftskip-\pshrinking}%

\newcommand{\aproblem}{%
  \setcounter{subprobcount}{0}%
  \stepcounter{probcount}%
  \def\@currentlabel{\arabic{probcount}}%
  \ifhmode
    \unskip \par
  \fi
%  \addpenalty{-4000}%
  \iffirstprob\else\addvspace\probsep\fi
  \firstprobfalse
  \hskip -\labelwidth\hskip -\labelsep 
  \hbox to\labelwidth{\hss\textbf{\arabic{probcount}.}}\hskip\labelsep
}%

\newcommand{\subprob}{\item\def\@currentlabel{\arabic{probcount}\alph{\thelistlabel}}}
\newcommand{\skipproblem}{\stepcounter{probcount}}


%% The following commands put defined left and right headers on the top, and a page number
%% on the bottom of all pages beyond page 1
\usepackage{fancyhdr}
\pagestyle{fancy}
\fancyfoot[C]{\ifnum \value{page} > 1\relax\thepage\fi}
\fancyhead[L]{\ifx\@doclabel\@empty\else\@doclabel\fi}
\fancyhead[R]{\ifx\@docdate\@empty\else\@docdate\fi}
\headheight 15pt
\def\doclabel#1{\gdef\@doclabel{#1}}
\def\docdate#1{\gdef\@docdate{#1}}
\makeatother

%% General formatting parameters
\parindent 0pt
\parskip 6pt plus 1pt


\doclabel{Math F251: Sections 3.7 and 3.6 Worksheet}
\docdate{Friday 1 March 2019}


\begin{document}
\renewcommand{\d}{\displaystyle}

\begin{aproblems}
% 3.7 #20
\aproblem Newton's Law of Gravitation says that the magnitude $F$ of the force exerted by a body of mass $m$ on a body of mass $M$ is 
    $$F = \frac{G m M}{r^2}$$
where $G$ is the gravitational constant and $r$ is the distance between the bodies.
\renewcommand{\labelenumi}{(\alph{enumi})}
\begin{enumerate}
\item Find $dF/dr$ and explain its meaning.  What does the minus indicate?
\vfill

\item Assume we measure mass in kilograms, distance in meters, and force in Newtons.  What are the units of $dF/dr$?
\vspace{0.5in}

\item Find $dF/dm$ and explain its meaning and units.
\vfill
\end{enumerate}

% 3.7 #18
\aproblem  A tank holds 5000 gallons of water which drains from the bottom of the tank in 40 minutes.  The volume of water remaining in the tank after $t$ minutes is
    $$V = 5000 \left(1-\frac{1}{40} t\right)^2$$
for $0\le t\le 40$.  Find the rate at which water is draining from the tank after (a) 5 min, (b) 20 min, and (c) 40 min.  Which is fastest/slowest?
\vspace{2.5in}

\clearpage\newpage
% 3.6 #6, 21, uses arcsin, one more
\aproblem  Differentiate the functions.
\begin{itemize}
\item[]
    $$y = \frac{1}{\log_3 x} \hspace{5.0in}$$
\vfill

\item[]
    $$y = \tan\left[\ln(ax+b)\right] \hspace{4.5in}$$
\vfill

\item[]
    $$H(z) = 7^z \arctan z \hspace{4.5in}$$
\vfill

\item[]
    $$g(t) = \frac{\ln t}{\arcsin(t^2)+1} \hspace{4.5in}$$
\vfill
\end{itemize}

\end{aproblems}

\end{document}
