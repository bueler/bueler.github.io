\documentclass[12pt]{article}
\usepackage[top=1in, bottom=0.7in, left=1.05in, right=1.05in]{geometry}

\usepackage{graphicx,color,enumerate,multicol}
\usepackage{amsmath,amsthm,amsbsy}
\usepackage{palatino}

%% Setup aproblem environment, 
%% aproblem items
%% subproblems environment
%% subproblem items
\makeatletter
\newcounter{probcount}
\newcounter{subprobcount}
\newlength\probsep
\newlength\pshrinking
\newif\iffirstprob
\newenvironment{aproblems}%
  {\ifhmode\unskip\par\fi\setcounter{probcount}{0}\probsep\parskip
  \sbox\@tempboxa{\textbf{9.}}\pshrinking\wd\@tempboxa\advance\pshrinking\labelsep
  \let\hproblem\aproblem
  \advance\linewidth -\pshrinking
  \advance\@totalleftmargin\pshrinking
  \advance\leftskip\pshrinking}%
  {\ifhmode\unskip \par\fi\advance\leftskip-\pshrinking}%

\newcommand{\aproblem}{%
  \setcounter{subprobcount}{0}%
  \stepcounter{probcount}%
  \def\@currentlabel{\arabic{probcount}}%
  \ifhmode
    \unskip \par
  \fi
%  \addpenalty{-4000}%
  \iffirstprob\else\addvspace\probsep\fi
  \firstprobfalse
  \hskip -\labelwidth\hskip -\labelsep 
  \hbox to\labelwidth{\hss\textbf{\arabic{probcount}.}}\hskip\labelsep
}%

\newcommand{\subprob}{\item\def\@currentlabel{\arabic{probcount}\alph{\thelistlabel}}}
\newcommand{\skipproblem}{\stepcounter{probcount}}


%% The following commands put defined left and right headers on the top, and a page number
%% on the bottom of all pages beyond page 1
\usepackage{fancyhdr}
\pagestyle{fancy}
\fancyfoot[C]{\ifnum \value{page} > 1\relax\thepage\fi}
\fancyhead[L]{\ifx\@doclabel\@empty\else\@doclabel\fi}
\fancyhead[R]{\ifx\@docdate\@empty\else\@docdate\fi}
\headheight 15pt
\def\doclabel#1{\gdef\@doclabel{#1}}
\def\docdate#1{\gdef\@docdate{#1}}
\makeatother

%% General formatting parameters
\parindent 0pt
\parskip 6pt plus 1pt


\doclabel{Math F251: Midterm I Review Worksheet}
\docdate{Monday 11 February 2019}


\begin{document}
\renewcommand{\d}{\displaystyle}

The midterm will mostly cover Chapter 2, but Chapter 1 skills are needed at all times.  The questions will be like assigned problems in Chapter 2.  Note section 2.4 is skipped.

There are many old versions of Midterm Exam 1 on the ``Exam'' tab on the website.

Here are important topics which you should review and make sure you understand, with the sections where they appear.  Find and do example problems from each topic!
\begin{itemize}
\item average velocity and secant line slope \hfill \emph{\S 2.1}
\item definition of the basic (two-sided) limit $\lim_{x\to a} f(x)$ as a sentence \hfill\emph{\S 2.2}
\item one-sided limits  \hfill \emph{\S 2.2}
\item infinite limits  \hfill \emph{\S 2.2}
\item limits at infinity  \hfill \emph{\S 2.6}
\item vertical and horizontal asymptotes are defined by limits  \hfill \emph{\S 2.2, 2.6}
\item using values close to $x=a$ to estimate the limit  \hfill \emph{\S 2.2}
\item using algebra and limit laws to compute limits  \hfill \emph{\S 2.3, 2.6}
\item getting limits from a given graph \emph{or} using given limits and values to generate (sketch) a graph  \hfill \emph{\S 2.2, 2.3, 2.5, 2.6}
\item definition of continuity  \hfill \emph{\S 2.5}
\item common functions are continuous on their domains  \hfill \emph{\S 2.5}
\item using the Intermediate Value Theorem to show equations have solutions \hfill \emph{\S 2.5}
\item definition of the derivative as a limit  \hfill \emph{\S 2.7}
\item computing a derivative from the limit definition  \hfill \emph{\S 2.7, 2.8}
\item tangent line slope and instantaneous velocity: they are derivatives  \hfill \emph{\S 2.7}
\item find equation of a tangent line  \hfill \emph{\S 2.7}
\item the derivative as a new function derived from $f(x)$  \hfill \emph{\S 2.8}
\item sketching $f'(x)$ based on $f(x)$  \hfill \emph{\S 2.8}
\item notation: $f'(x)=y'=\frac{df}{dx}=\frac{dy}{dx}$  \hfill \emph{\S 2.8}
\item higher derivatives (and notation for them)  \hfill \emph{\S 2.8}
\end{itemize}

\vfill
The reverse side has a few example problems; it is a worksheet not a sample of the Midterm Exam.

\clearpage\newpage
\thispagestyle{empty}

\phantom{foo}\vspace{-0.7in}
\begin{aproblems}
\aproblem (\S 2.6 \#9) \quad  Sketch the graph of a function that satisfies all these conditions:
\small
    $$\hspace{-0.5in} f(0)=3, \, \lim_{x\to 0^-} f(x) = 4, \, \lim_{x\to 0^+} f(x) = 2, \, \lim_{x\to-\infty} f(x) = -\infty, \, \lim_{x\to 4^-} f(x) = -\infty, \, \lim_{x\to 4^+} f(x) = \infty, \, \lim_{x\to\infty} f(x) = 3$$
\normalsize
\vfill

\aproblem  Find $f'(x)$ using the definition if $f(x)=\sqrt{x}$.
\vfill

\aproblem (\S 2.7 \#7) \quad  Using the result of the last problem, find an equation of the tangent line to $y=\sqrt{x}$ at the point $(1,1)$.
\vfill

\aproblem (\S 2.6 \#50) \quad  Find the horizontal and vertical asymptotes of the curve, and state the limits which justify these asymptotes:
    $$y = \frac{1+x^4}{x^2-x^4} \hspace{4.5in}$$
\vfill

\aproblem (\S 2.3 \#49) \quad  Let $g(x) = \frac{x^2+x-6}{|x-2|}$.  (a) Find $\lim_{x\to 2^-} g(x)$ and $\lim_{x\to 2^+} g(x)$.  (b) Does $\lim_{x\to 2} g(x)$ exist?
\vfill

\end{aproblems}

\end{document}
