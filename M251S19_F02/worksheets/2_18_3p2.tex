\documentclass[12pt]{article}
\usepackage[top=1in, bottom=1in, left=1.25in, right=1.25in]{geometry}

\usepackage{graphicx,color,enumitem}
\usepackage{amsmath,amsthm,amsbsy}
\usepackage{palatino}

%% Setup aproblem environment, 
%% aproblem items
%% subproblems environment
%% subproblem items
\makeatletter
\newcounter{probcount}
\newcounter{subprobcount}
\newlength\probsep
\newlength\pshrinking
\newif\iffirstprob
\newenvironment{aproblems}%
  {\ifhmode\unskip\par\fi\setcounter{probcount}{0}\probsep\parskip
  \sbox\@tempboxa{\textbf{9.}}\pshrinking\wd\@tempboxa\advance\pshrinking\labelsep
  \let\hproblem\aproblem
  \advance\linewidth -\pshrinking
  \advance\@totalleftmargin\pshrinking
  \advance\leftskip\pshrinking}%
  {\ifhmode\unskip \par\fi\advance\leftskip-\pshrinking}%

\newcommand{\aproblem}{%
  \setcounter{subprobcount}{0}%
  \stepcounter{probcount}%
  \def\@currentlabel{\arabic{probcount}}%
  \ifhmode
    \unskip \par
  \fi
%  \addpenalty{-4000}%
  \iffirstprob\else\addvspace\probsep\fi
  \firstprobfalse
  \hskip -\labelwidth\hskip -\labelsep 
  \hbox to\labelwidth{\hss\textbf{\arabic{probcount}.}}\hskip\labelsep
}%

\newcommand{\subprob}{\item\def\@currentlabel{\arabic{probcount}\alph{\thelistlabel}}}
\newcommand{\skipproblem}{\stepcounter{probcount}}


%% The following commands put defined left and right headers on the top, and a page number
%% on the bottom of all pages beyond page 1
\usepackage{fancyhdr}
\pagestyle{fancy}
\fancyfoot[C]{\ifnum \value{page} > 1\relax\thepage\fi}
\fancyhead[L]{\ifx\@doclabel\@empty\else\@doclabel\fi}
\fancyhead[R]{\ifx\@docdate\@empty\else\@docdate\fi}
\headheight 15pt
\def\doclabel#1{\gdef\@doclabel{#1}}
\def\docdate#1{\gdef\@docdate{#1}}
\makeatother

%% General formatting parameters
\parindent 0pt
\parskip 6pt plus 1pt


\doclabel{Math F251: Sections 3.2 and 3.3 Worksheet}
\docdate{Monday 18 February 2019}


\begin{document}
\renewcommand{\d}{\displaystyle}

\begin{aproblems}
% like 3.2 #1
\aproblem  Find the derivative of $f(x) = (x + x^2) (x^{-1} + 3)$ in two ways:
\renewcommand{\labelenumi}{(\roman{enumi})}
\begin{enumerate}
\item by the product rule:

\vfill
\item by first expanding the product:

\vfill
\end{enumerate}

% 3.2 # 14, 4 (like), 26
\aproblem  Differentiate.
\renewcommand{\labelenumi}{(\alph{enumi})}
\begin{enumerate}
\item \quad $\d y = \frac{\sqrt{x}}{2+x}$

\vfill
\item \quad $\d g(x) = (\pi^{1/2} + 5\sqrt{x})\, e^x$

\vfill
\item \quad $\d f(x) = \frac{ax + b}{cx + d}$
\end{enumerate}

\vfill

\clearpage
\newpage
% 3.2 #46
\aproblem If $h(2)=4$ and $h'(2)=-3$, find
    $$\frac{d}{dx}\left(\frac{h(x)}{x}\right)\Big|_{x=2}$$ 
\vfill

% 3.3 #17
\aproblem  Consider these facts:
\begin{itemize}[itemsep=-1mm]
\item $\csc x = 1/\sin x$
\item $\cot x = \cos x/\sin x$
\item $(\sin x)' = \cos x$
\end{itemize}
Use the quotient rule and the above facts to show that
    $$\frac{d}{dx} \left(\csc x\right) = - \csc x \cot x$$
\vfill

% 3.3 #15
\aproblem  Differentiate $f(\theta) = \theta \cos \theta \sin \theta$.
\vfill
\end{aproblems}

\end{document}
