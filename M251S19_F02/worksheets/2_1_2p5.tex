\documentclass[12pt]{article}
\usepackage[top=1in, bottom=1in, left=1.25in, right=1.25in]{geometry}

\usepackage{graphicx,color,enumerate,multicol}
\usepackage{amsmath,amsthm,amsbsy}
\usepackage{palatino}

%% Setup aproblem environment, 
%% aproblem items
%% subproblems environment
%% subproblem items
\makeatletter
\newcounter{probcount}
\newcounter{subprobcount}
\newlength\probsep
\newlength\pshrinking
\newif\iffirstprob
\newenvironment{aproblems}%
  {\ifhmode\unskip\par\fi\setcounter{probcount}{0}\probsep\parskip
  \sbox\@tempboxa{\textbf{9.}}\pshrinking\wd\@tempboxa\advance\pshrinking\labelsep
  \let\hproblem\aproblem
  \advance\linewidth -\pshrinking
  \advance\@totalleftmargin\pshrinking
  \advance\leftskip\pshrinking}%
  {\ifhmode\unskip \par\fi\advance\leftskip-\pshrinking}%

\newcommand{\aproblem}{%
  \setcounter{subprobcount}{0}%
  \stepcounter{probcount}%
  \def\@currentlabel{\arabic{probcount}}%
  \ifhmode
    \unskip \par
  \fi
%  \addpenalty{-4000}%
  \iffirstprob\else\addvspace\probsep\fi
  \firstprobfalse
  \hskip -\labelwidth\hskip -\labelsep 
  \hbox to\labelwidth{\hss\textbf{\arabic{probcount}.}}\hskip\labelsep
}%

\newcommand{\subprob}{\item\def\@currentlabel{\arabic{probcount}\alph{\thelistlabel}}}
\newcommand{\skipproblem}{\stepcounter{probcount}}


%% The following commands put defined left and right headers on the top, and a page number
%% on the bottom of all pages beyond page 1
\usepackage{fancyhdr}
\pagestyle{fancy}
\fancyfoot[C]{\ifnum \value{page} > 1\relax\thepage\fi}
\fancyhead[L]{\ifx\@doclabel\@empty\else\@doclabel\fi}
\fancyhead[R]{\ifx\@docdate\@empty\else\@docdate\fi}
\headheight 15pt
\def\doclabel#1{\gdef\@doclabel{#1}}
\def\docdate#1{\gdef\@docdate{#1}}
\makeatother

%% General formatting parameters
\parindent 0pt
\parskip 6pt plus 1pt


\doclabel{Math F251: Sections 2.5 and 2.6 Worksheet}
\docdate{1 February 2019}


\begin{document}
\renewcommand{\d}{\displaystyle}

% 2.5 #40, 58, 73 and 2.6 #8, 49
% 2.6 first

\begin{aproblems}

\aproblem  Sketch the graph of a function that satisfies all of the given conditions:
\begin{itemize}
\item $\lim_{x\to\infty} f(x) = 3$
\item $\lim_{x\to 2^-} f(x) = \infty$
\item $\lim_{x\to 2^+} f(x) = -\infty$
\item $f$ is odd
\end{itemize}

\vfill

\aproblem  Find all the vertical and horizontal asymptotes of the graph
    $$y = \frac{2x^2 + x - 1}{x^2 + x - 2},$$
and clearly state limits which justify these asymptotes.  (\emph{Also make a rough sketch of the graph.  You may be able to confirm your work by graphing calculator.})
\vfill

\clearpage \newpage
\aproblem  Show that $f$ is continuous on $(-\infty,\infty)$, and sketch the graph:
\begin{flalign*}
    f(x) &= \begin{cases} \sin x & \text{ if } x < \pi/4 \\
                          \cos x & \text{ if } x \ge \pi/4 \end{cases} &
\end{flalign*}

\vfill

\aproblem  Prove that the equation has at least one real root:
    $$\ln x = 3 - 2x$$
\noindent (\emph{A calculator can help find an accurate approximation, but this is not required!})

\vspace{1.5in}

\aproblem \emph{A challenge problem, but actually easy.  It follows from the Intermediate Value Theorem.  Start by sketching elevation versus time for each day, one on top of the other.}

A Tibetan monk leaves the monastery at 7:00 \textsc{am} and takes his usual parth to the top of the mountain, arriving at 7:00 \textsc{pm} and sleeping on top.  The next morning he starts at 7:00 \textsc{am} at the top and takes the same path back, arriving at the monastery at 7:00 \textsc{am}.  Show that there is a point on the path that the monk will cross at exactly the same time of day on both days.

\vfill

\end{aproblems}

\end{document}
