\documentclass[12pt]{article}
\usepackage[top=0.9in, bottom=0.9in, left=0.9in, right=1.1in]{geometry}

\usepackage{graphicx,color,enumitem}
\usepackage{amsmath,amsthm,amsbsy}
\usepackage{palatino}

\usepackage{tikz}

%% Setup aproblem environment, 
%% aproblem items
%% subproblems environment
%% subproblem items
\makeatletter
\newcounter{probcount}
\newcounter{subprobcount}
\newlength\probsep
\newlength\pshrinking
\newif\iffirstprob

\newenvironment{aproblems}%
  {\ifhmode\unskip\par\fi\setcounter{probcount}{0}\probsep\parskip
  \sbox\@tempboxa{\textbf{9.}}\pshrinking\wd\@tempboxa\advance\pshrinking\labelsep
  \let\hproblem\aproblem
  \advance\linewidth -\pshrinking
  \advance\@totalleftmargin\pshrinking
  \advance\leftskip\pshrinking}%
  {\ifhmode\unskip \par\fi\advance\leftskip-\pshrinking}%

\newcommand{\aproblem}{%
  \setcounter{subprobcount}{0}%
  \stepcounter{probcount}%
  \def\@currentlabel{\arabic{probcount}}%
  \ifhmode
    \unskip \par
  \fi
%  \addpenalty{-4000}%
  \iffirstprob\else\addvspace\probsep\fi
  \firstprobfalse
  \hskip -\labelwidth\hskip -\labelsep 
  \hbox to\labelwidth{\hss\textbf{\arabic{probcount}.}}\hskip\labelsep
}%


%% The following commands put defined left and right headers on the top, and a page number
%% on the bottom of all pages beyond page 1
\usepackage{fancyhdr}
\pagestyle{fancy}
\fancyfoot[C]{\ifnum \value{page} > 1\relax\thepage\fi}
\fancyhead[L]{\ifx\@doclabel\@empty\else\@doclabel\fi}
\fancyhead[R]{\ifx\@docdate\@empty\else\@docdate\fi}
\headheight 15pt
\def\doclabel#1{\gdef\@doclabel{#1}}
\def\docdate#1{\gdef\@docdate{#1}}
\makeatother

%% General formatting parameters
\parindent 0pt
\parskip 6pt plus 1pt


\doclabel{Math F251: constructing exam problems, and review problems}
\docdate{Monday 29 April 2019}


\begin{document}
\renewcommand{\d}{\displaystyle}

\begin{aproblems}
\aproblem (\emph{See examples and exercises in} \S 2.2 and \S 2.6.)  Give an example of a graph $y=f(x)$ with a vertical asymptote at $x=-1$ and a horizontal asymptote at $y=2$.
\vfill

\aproblem (\emph{See} \S 3.4.)  Build an example of a complicated chain rule derivative question.  That is, write down $f(x)$ and compute the derivative $f'(x)$.
\vspace{1.5in}

\aproblem (\emph{See} \S 5.5.)  Write the previous example as an indefinite integration question.  Give a substitution which will solve it.  Complete the integration.
\vspace{2.0in}

\hrulefill

Some advice for the actual Final Exam: \qquad \large 

\centerline{\textbf{Read the question.  Don't just guess it is of a certain type.}}

\normalsize

\newpage
\aproblem (\emph{See} \S 4.7.)  A steel cylindrical can is to hold 1 L of oil.  Find the dimensions of the can that will minimize the amount of steel.
\vfill

\aproblem (\emph{See} \S 5.1 and \S 5.2.)  For the integral $\displaystyle \int_0^6 \frac{1}{1+x^4}\,dx$, compute the Riemann sums with $n=3$ rectangles and both left and right endpoints.
\vspace{2.0in}

\newpage
\aproblem (\emph{See} \S 4.3 and \S 4.5.)  Find the critical points, intervals of increase and decrease, and points of inflection of $f(x)=x^3-3x-1$.  Then sketch the graph $y=f(x)$.
\vfill

\aproblem (\emph{See} \S 4.8.)  In the graph above there is a solution of $f(x)=0$ near $x=2$.  Approximate it using one step of Newton's method, and add that to your sketch.
\vfill

\aproblem (\emph{See} \S 3.5.)  Find $dy/dx$ by implicit differentiation: \qquad $y \cos x = x^2 + y^2$
\vspace{2.0in}

\end{aproblems}

\end{document}
