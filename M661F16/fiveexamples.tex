\documentclass[12pt]{amsart}
%prepared in AMSLaTeX, under LaTeX2e
\addtolength{\oddsidemargin}{-.55in} 
\addtolength{\evensidemargin}{-.55in}
\addtolength{\topmargin}{-.2in}
\addtolength{\textwidth}{1.0in}
\addtolength{\textheight}{.4in}

\renewcommand{\baselinestretch}{1.05}

\usepackage{verbatim,fancyvrb}

\usepackage{tikz,palatino,amssymb}

\newtheorem*{thm}{Theorem 16.0}
\newtheorem*{defn}{Definition}
\newtheorem*{example}{Example}
\newtheorem*{problem}{Problem}
\newtheorem*{remark}{Remark}

\newcommand{\mtt}{\texttt}
\usepackage{alltt,xspace}
\newcommand{\mfile}[1]
{\medskip\begin{quote}\scriptsize \begin{alltt}\input{#1.m}\end{alltt} \normalsize\end{quote}\medskip}

%\usepackage[final]{graphicx}

\usepackage[pdftex, colorlinks=true, plainpages=false, linkcolor=blue, citecolor=red, urlcolor=blue]{hyperref}

% macros
\newcommand{\br}{\mathbf{r}}
\newcommand{\bv}{\mathbf{v}}
\newcommand{\bx}{\mathbf{x}}
\newcommand{\by}{\mathbf{y}}

\newcommand{\CC}{\mathbb{C}}
\newcommand{\RR}{\mathbb{R}}
\newcommand{\ZZ}{\mathbb{Z}}

\newcommand{\eps}{\epsilon}
\newcommand{\grad}{\nabla}
\newcommand{\lam}{\lambda}
\newcommand{\lap}{\triangle}

\newcommand{\ip}[2]{\ensuremath{\left<#1,#2\right>}}

%\renewcommand{\det}{\operatorname{det}}
\newcommand{\onull}{\operatorname{null}}
\newcommand{\rank}{\operatorname{rank}}
\newcommand{\range}{\operatorname{range}}

\newcommand{\prob}[1]{\bigskip\noindent\textbf{#1.}\quad }
\newcommand{\exer}[2]{\prob{Exercise #2 in Lecture #1}}

\newcommand{\pts}[1]{(\emph{#1 pts}) }
\newcommand{\epart}[1]{\medskip\noindent\textbf{(#1)}\quad }
\newcommand{\ppart}[1]{\,\textbf{(#1)}\quad }

\newcommand{\Matlab}{\textsc{Matlab}\xspace}

\DefineVerbatimEnvironment{mVerb}{Verbatim}{numbersep=2mm,
frame=lines,framerule=0.1mm,framesep=2mm,xleftmargin=4mm,fontsize=\footnotesize}

\newcommand{\ema}{\emach}
\newcommand{\emach}{\eps_{\!_{\text{m}}}}


\begin{document}
\scriptsize \noindent Math 661 Optimization (Bueler) \hfill 23 August, 2016
\normalsize

\medskip\bigskip
\Large
\centerline{Five example optimization problems}

\bigskip\medskip
\normalsize

\thispagestyle{empty}

I am starting this course with five examples.  The textbook\footnote{Nocedal \& Wright, \emph{Numerical Optimization}, 2nd ed., 2006} does not give any such fleshed-out examples, but all experts in this field have such examples in their backgrounds.

There are three goals in starting this way:
\renewcommand{\labelenumi}{\arabic{enumi})}
\begin{enumerate}
\item To suggest how optimization can come from real-world applications.
\item To allow you to create, for yourself, some of the basic ideas for how to solve these problems numerically.
\item To provide examples on which to practice, or learn, some \Matlab\footnote{Or Octave, which runs the same programs.  Or you can use other languages such as Python.} programming.
\end{enumerate}
On goal 2), \emph{the method you create may be ``brute force'' or inefficient, but that is just fine!}

Assignment \#1 will be based on these examples.  Later Assignments will be largely based on textbook exercises, but I'll add my own problems and examples when needed.

Note that each example has a ``code name'' like ``\texttt{calcone}'' which I will use to name the corresponding \Matlab code (e.g.~\texttt{calcone.m}) when I hand out solutions to Assignment \# 1.

\bigskip
\renewcommand{\labelenumi}{\Roman{enumi}. \quad}
\begin{enumerate}
\item (\texttt{calcone})  \quad Let
    $$f(x) = \left(x^2 + \cos x\right)^2 - 10 \sin(5 x).$$
Compute
    $$\min_{x\in [0,2]} f(x)$$

You may write the same problem as an inequality-constrained optimization in form (1.1), on page 3 of the textbook, namely as
	$$\min_{x\in\RR} f(x) \qquad \text{subject to }\quad \begin{matrix} x \ge 0 \\ 2 - x \ge 0\end{matrix}$$
In this standard form, $\mathcal{E}=\emptyset$, $\mathcal{I}=\{1,2\}$, $c_1(x)=x$, and $c_2(x)=2-x$.  However, I would not assert that there is any particular benefit to this standardized form in this case.

You saw this kind of problem in Calculus I.  This one is hard to do by hand so it will benefit from some programming and visualization using \Matlab.  Start your solution by plotting $f(x)$ on the given interval; you'll know something close to the solution just from that picture.


\medskip
\item (\texttt{fit})  \quad 

\medskip
\item (\texttt{salmon})  \quad 

\medskip
\item (\texttt{tsp})  \quad 

\medskip
\item (\texttt{beam})  \quad 
\end{enumerate}

\end{document}

