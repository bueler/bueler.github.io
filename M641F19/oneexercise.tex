\documentclass[12pt]{amsart}
%prepared in AMSLaTeX, under LaTeX2e

\theoremstyle{definition}
\newtheorem*{defn}{Definition}

\theoremstyle{plain}
\newtheorem*{lem}{Lemma}
\newtheorem*{prop}{Proposition}
\newtheorem*{thm}{Theorem}

\theoremstyle{remark}
\newtheorem*{example}{Example}
\newtheorem*{remark}{Remark}

\newcommand{\CC}{\mathbb{C}}
\newcommand{\NN}{\mathbb{N}}
\newcommand{\QQ}{\mathbb{Q}}
\newcommand{\RR}{\mathbb{R}}
\newcommand{\ZZ}{\mathbb{Z}}

\newcommand{\eps}{\epsilon}


\begin{document}
\noindent \footnotesize 26 August 2019  % REPLACE HERE
\normalsize \hfill Ed Bueler  % REPLACE HERE


\thispagestyle{empty}  % no need for a page number on this page

\bigskip

\Large\centerline{\textbf{Exercise 0 in Chapter 0}}  % REPLACE HERE
\normalsize


\bigskip
% REPLACE EVERYTHING BELOW HERE:
You can write text to explain something, if needed, but you should write in complete sentences.  Here we need a definition.

\begin{defn} A real number $x$ is \emph{irrational} if $x\notin \QQ$.
\end{defn}

Lemmas, propositions, and theorems do not need to be numbered, but please \emph{do} use the theorem and proof environments as shown here.

\begin{thm} $\sqrt{2}$ is irrational.
\end{thm}

\begin{proof} Suppose $x=\sqrt{2} \in \QQ$.  Then there exist $m,n\in\ZZ$ so that $x=m/n$ and $m,n$ have no common factors.  But then $2=m^2/n^2$ so $m^2=2n^2$.  Since $m^2$ has a factor of $2$, and because $2$ is prime, it follows that $2|m$, so $m=2k$ for some $k\in\ZZ$.  Now $(2k)^2=2n^2$ or $2k^2=n^2$ after simplification.  Since $n^2$ has a factor of $2$ it follows that $2|n$, but that means both $m$ and $n$ had a factor of $2$.  This contradicts the assumption that $x$ was rational (and was represented without common factors).
\end{proof}

% EXCEPT LEAVE THIS:
\end{document}
