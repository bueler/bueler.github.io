\documentclass[11pt]{article}
\usepackage[top=0.8in, bottom=0.7in, left=0.7in, right=0.9in]{geometry}

\usepackage{graphicx,color,enumitem}
\usepackage{amsmath,amsthm,amsbsy}
\usepackage{palatino}

\usepackage{tikz}

%% Setup aproblem environment, 
%% aproblem items
%% subproblems environment
%% subproblem items
\makeatletter
\newcounter{probcount}
\newcounter{subprobcount}
\newlength\probsep
\newlength\pshrinking
\newif\iffirstprob
\newenvironment{aproblems}%
  {\ifhmode\unskip\par\fi\setcounter{probcount}{0}\probsep\parskip
  \sbox\@tempboxa{\textbf{9.}}\pshrinking\wd\@tempboxa\advance\pshrinking\labelsep
  \let\hproblem\aproblem
  \advance\linewidth -\pshrinking
  \advance\@totalleftmargin\pshrinking
  \advance\leftskip\pshrinking}%
  {\ifhmode\unskip \par\fi\advance\leftskip-\pshrinking}%

\newcommand{\aproblem}{%
  \setcounter{subprobcount}{0}%
  \stepcounter{probcount}%
  \def\@currentlabel{\arabic{probcount}}%
  \ifhmode
    \unskip \par
  \fi
%  \addpenalty{-4000}%
  \iffirstprob\else\addvspace\probsep\fi
  \firstprobfalse
  %\hskip -\labelwidth\hskip -\labelsep 
  \hskip 0pt%-\labelsep 
  \hbox to\labelwidth{\hss\textbf{Example \arabic{probcount}.} \,}\hskip\labelsep
}%

\newcommand{\subprob}{\item\def\@currentlabel{\arabic{probcount}\alph{\thelistlabel}}}
\newcommand{\skipproblem}{\stepcounter{probcount}}


%% The following commands put defined left and right headers on the top, and a page number
%% on the bottom of all pages beyond page 1
\usepackage{fancyhdr}
\pagestyle{fancy}
\fancyfoot[C]{\ifnum \value{page} > 1\relax\thepage\fi}
\fancyhead[L]{\ifx\@doclabel\@empty\else\@doclabel\fi}
\fancyhead[R]{\ifx\@docdate\@empty\else\@docdate\fi}
\headheight 15pt
\def\doclabel#1{\gdef\@doclabel{#1}}
\def\docdate#1{\gdef\@docdate{#1}}
\makeatother

%% General formatting parameters
\parindent 0pt
\parskip 6pt plus 0pt


\doclabel{Math F251: Section 4.8 Handout}
\docdate{Thursday 8 November 2018}


\begin{document}
\renewcommand{\d}{\displaystyle}

\begin{aproblems}

\phantom{foo}

\vspace{-10mm}
Newton's method solves a problem
    $$f(x)=0$$
by starting with an initial estimate $x_0$.  It gets a new estimate from the old estimate by finding where the linearization $L(x) \approx f(x)$ crosses the $x$-axis.  (\emph{To understand this, write down the linearization at $a=x_0$.  Solve for $x$.})  So it uses this formula:
    $$x_{n+1} = x_n - \frac{f(x_n)}{f'(x_n)}.$$

\aproblem  Suppose you want to solve the equation
    $$3x^4-x^3+6x^2+4x-2 = 0.$$
I do not see how to factor this by hand.  (\emph{Do you?})  If we name the left side of the equation ``$f(x)$'' and if we go looking for a root of $f(x)$ then we might notice $f(0)=-2$ and $f(1)=10$.  Because $f$ is continuous, by the intermediate value theorem there is a root on $(0,1)$.  Also, the derivative is an easy calculation: $f'(x) = 12 x^3 - 3 x^2 + 12 + 4.$

Newton's method with initial estimate $x_0=1/2$ gives the next value by
    $$x_1 = x_0 - \frac{f(x_0)}{f'(x_0)} = 0.354651162790698$$
(\emph{It is nice to immediately use a \emph{serious} calculator, namely one with a big screen and the ability to run a loop.  Thus I used \textsc{Matlab} on my laptop.})  The next four iterations gave consistent numbers which strongly suggest $x=1/3$ is an exact root:
\begin{align*}
x_0 &= 0.500000000000000 \\
x_1 &= \underline{0.3}54651162790698 \\
x_2 &= \underline{0.333}718551391767 \\
x_3 &= \underline{0.333333}461355651 \\
x_4 &= \underline{0.3333333333333}47 \\
x_5 &= \underline{0.333333333333333}
\end{align*}
Notice how the number of correct digits doubles from $x_1$ to $x_4$.  It stops doubling only because the computer only stores about 16 decimal digits.  In fact we have discovered how to factor the function: $f(x)=(x-\frac{1}{3}) (3x^3+6x+6)$.

\medskip
\aproblem  Suppose we want to find the maximum of $g(x) = x^2 - 2 \sin x$ on the interval $[0,1]$.  We take the derivative and set it to zero to find the critical numbers:
    $$2x - 2 \cos x=0$$
You can factor and remove the ``$2$,'' but otherwise this is an equation I do not know how to solve by hand.  So I try Newton's method.  Let $f(x)=x-\cos x$ so $f'(x)=1+\sin(x)$.  Newton's method starting with $x_0=0.5$ gives these five iterates, and the last two agree to all digits:
\begin{align*}
x_0 &= 0.500000000000000 \\
x_1 &= \underline{0.7}55222417105636 \\
x_2 &= \underline{0.739}141666149879 \\
x_3 &= \underline{0.739085133}920807 \\
x_4 &= \underline{0.739085133215161} \\
x_5 &= \underline{0.739085133215161}
\end{align*}
In fact $f(x_4)=0$ to the accuracy of the computer.  Again notice the approximate doubling of the number of correct digits at each Newton step.

\end{aproblems}

\end{document}
