\documentclass[12pt]{article}

\usepackage{graphicx,color,enumerate,multicol}
\usepackage[top=1in, bottom=1in, left=1.25in, right=1.25in]{geometry}

%% Use Minion fonts if available.  Otherwise Times.
\IfFileExists{MinionPro.sty}{\usepackage[lf]{MinionPro}}{}
\usepackage{amsmath,amsthm,amsbsy}
\IfFileExists{MinionPro.sty}{}{\usepackage{times,txfonts}}

%% Setup aproblem environment, 
%% aproblem items
%% subproblems environment
%% subproblem items
\makeatletter
\newcounter{probcount}
\newcounter{subprobcount}
\newlength\probsep
\newlength\pshrinking
\newif\iffirstprob
\newenvironment{aproblems}%
  {\ifhmode\unskip\par\fi\setcounter{probcount}{0}\probsep\parskip
  \sbox\@tempboxa{\textbf{9.}}\pshrinking\wd\@tempboxa\advance\pshrinking\labelsep
  \let\hproblem\aproblem
  \advance\linewidth -\pshrinking
  \advance\@totalleftmargin\pshrinking
  \advance\leftskip\pshrinking}%
  {\ifhmode\unskip \par\fi\advance\leftskip-\pshrinking}%

\newcommand{\aproblem}{%
  \setcounter{subprobcount}{0}%
  \stepcounter{probcount}%
  \def\@currentlabel{\arabic{probcount}}%
  \ifhmode
    \unskip \par
  \fi
%  \addpenalty{-4000}%
  \iffirstprob\else\addvspace\probsep\fi
  \firstprobfalse
  \hskip -\labelwidth\hskip -\labelsep 
  \hbox to\labelwidth{\hss\textbf{\arabic{probcount}.}}\hskip\labelsep
}%

\newcommand{\subprob}{\item\def\@currentlabel{\arabic{probcount}\alph{\thelistlabel}}}
\newcommand{\skipproblem}{\stepcounter{probcount}}


%% The following commands put defined left and right headers on the top, and a page number
%% on the bottom of all pages beyond page 1
\usepackage{fancyhdr}
\pagestyle{fancy}
\fancyfoot[C]{\ifnum \value{page} > 1\relax\thepage\fi}
\fancyhead[L]{\ifx\@doclabel\@empty\else\@doclabel\fi}
\fancyhead[R]{\ifx\@docdate\@empty\else\@docdate\fi}
\headheight 15pt
\def\doclabel#1{\gdef\@doclabel{#1}}
\def\docdate#1{\gdef\@docdate{#1}}
\makeatother

%% General formatting parameters
\parindent 0pt
\parskip 6pt plus 1pt

\doclabel{Math F251: Section 2.1 Activity (Worksheet)}
\docdate{5 September 2018}

\begin{document}
%\renewcommand{\d}{\displaystyle}

\begin{aproblems}
\aproblem Here is a table of the temperature anomaly data, for recent years, from NASA.  I showed the whole 1880--2017 data set in class.  The first column is the year.  The second column is the difference of the globally-averaged temperature for that year minus the average of the 1951--1980 period, in Celsius.  The plot below shows this data.

\bigskip
\hspace{-15mm}\begin{minipage}[b]{0.4\textwidth}
\footnotesize
\begin{verbatim}
        1990        0.44
        1991        0.41
        1992        0.22
        1993        0.24
        1994        0.31
        1995        0.44
        1996        0.33
        1997        0.47
        1998        0.62
        1999         0.4
        2000         0.4
        2001        0.54
        2002        0.62
        2003        0.61
        2004        0.53
        2005        0.67
        2006        0.62
        2007        0.64
        2008        0.52
        2009        0.63
        2010         0.7
        2011        0.57
        2012        0.61
        2013        0.64
        2014        0.73
        2015        0.86
        2016        0.99
        2017         0.9
\end{verbatim}
\end{minipage}
\begin{minipage}[b]{0.6\textwidth}
\qquad \includegraphics[width=0.95\textwidth]{recentyeartemp}
\end{minipage}

\normalsize
Compute from the data:

\vspace{-5mm}
\begin{enumerate}
\item the average rate of change of temperature (i.e.~slope of the secant line) in the period 1990--2017

\bigskip
\item the highest average rate of change you can compute for a ten-year period

\bigskip
\item the lowest rate of change you can compute for a ten-year period

\bigskip
\item your estimate of the rate of change in the year 2010
\end{enumerate}

\vfill
\footnotesize
\emph{This example shows that slopes can always be computed, \emph{but} that noisy data does not really have a slope when you look at a small period.  See the next page for better-behaved functions.  Math 251 Calculus I will be \emph{entirely} about well-behaved functions.  You see, Calc I is not real life.}
\vfill

\clearpage \newpage
\aproblem 
\vfill


\end{aproblems}

\end{document}