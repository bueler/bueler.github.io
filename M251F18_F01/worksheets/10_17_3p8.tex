\documentclass[12pt]{article}
\usepackage[top=0.9in, bottom=0.9in, left=0.9in, right=1.1in]{geometry}

\usepackage{graphicx,color,enumitem}
\usepackage{amsmath,amsthm,amsbsy}
\usepackage{palatino}

%% Setup aproblem environment, 
%% aproblem items
%% subproblems environment
%% subproblem items
\makeatletter
\newcounter{probcount}
\newcounter{subprobcount}
\newlength\probsep
\newlength\pshrinking
\newif\iffirstprob
\newenvironment{aproblems}%
  {\ifhmode\unskip\par\fi\setcounter{probcount}{0}\probsep\parskip
  \sbox\@tempboxa{\textbf{9.}}\pshrinking\wd\@tempboxa\advance\pshrinking\labelsep
  \let\hproblem\aproblem
  \advance\linewidth -\pshrinking
  \advance\@totalleftmargin\pshrinking
  \advance\leftskip\pshrinking}%
  {\ifhmode\unskip \par\fi\advance\leftskip-\pshrinking}%

\newcommand{\aproblem}{%
  \setcounter{subprobcount}{0}%
  \stepcounter{probcount}%
  \def\@currentlabel{\arabic{probcount}}%
  \ifhmode
    \unskip \par
  \fi
%  \addpenalty{-4000}%
  \iffirstprob\else\addvspace\probsep\fi
  \firstprobfalse
  \hskip -\labelwidth\hskip -\labelsep 
  \hbox to\labelwidth{\hss\textbf{\arabic{probcount}.}}\hskip\labelsep
}%

\newcommand{\subprob}{\item\def\@currentlabel{\arabic{probcount}\alph{\thelistlabel}}}
\newcommand{\skipproblem}{\stepcounter{probcount}}


%% The following commands put defined left and right headers on the top, and a page number
%% on the bottom of all pages beyond page 1
\usepackage{fancyhdr}
\pagestyle{fancy}
\fancyfoot[C]{\ifnum \value{page} > 1\relax\thepage\fi}
\fancyhead[L]{\ifx\@doclabel\@empty\else\@doclabel\fi}
\fancyhead[R]{\ifx\@docdate\@empty\else\@docdate\fi}
\headheight 15pt
\def\doclabel#1{\gdef\@doclabel{#1}}
\def\docdate#1{\gdef\@docdate{#1}}
\makeatother

%% General formatting parameters
\parindent 0pt
\parskip 6pt plus 1pt


\doclabel{Math F251: Sections 3.8 and 3.9 Worksheet}
\docdate{Wednesday 17 October 2018}


\begin{document}
\renewcommand{\d}{\displaystyle}

\begin{aproblems}
% 3.8 #19
\aproblem  The rate of change of atmospheric pressure $P$ with respect to altitude $h$ is proportional to $P$.  (This assumes the temperature is constant; let us assume that.)

\renewcommand{\labelenumi}{(\alph{enumi})}
\begin{enumerate}
\item  Write a differential equation corresponding to the first sentence above; use $k$ for the constant of proportionality.  Then write a formula for $P(h)$ in terms of $P(0)$, $k$, and $h$.

\vfill
\item  At a temperature of $15\,^\circ C$, the pressure is $101.3$ kPa at sea level and the pressure is $87.14$ kPa at $h=1000$ m.  From these facts, determine $P(0)$ and $k$.

\vfill
\item  What is the pressure at the top of Denali, at an altitude of $6187$ m?  (\emph{The problem in the book, \#19 in \S 3.8, has an error.  It calls it ``Mount McKinley.''})

\vfill
\item  At what altitude is the pressure $1/3$ of what it is at sea level?

\vfill
\end{enumerate}

\clearpage
\newpage
\thispagestyle{plain}

% 3.9 but I made it up
\aproblem  Gravel can be made by crushing rock and then running it through a screen for sorting.  Typically the sorted gravel is piled into a cone by a conveyor belt.  Because the gravel slides down the sides as the pile steepens, the sides alway have about the same angle (the \emph{angle of repose}) and the pile keeps its shape as it grows.

\renewcommand{\labelenumi}{(\alph{enumi})}
\begin{enumerate}
\item  Draw a conveyor belt feeding a conical pile of gravel.  Label the radius of the base of the cone as $r$ and its height as $h$.

\vfill
\item  The volume of a cone is
    $$V = \frac{1}{3} \pi r^2 h.$$
As the pile grows, which of the variables in this equation depend on time?

\vfill
\item  Compute $dV/dt$ by differentiating the above equation, keeping in mind that the other variables are also functions of time.

\vfill
\item  If the conveyor belt is adding $5 \,\text{m}^3/\text{min}$ of gravel to the pile, and the angle of the sides of the pile is $40^\circ$, at what rate is the height increasing when the base has radius $20$ m?
\vspace{2.5in}
\end{enumerate}

\end{aproblems}

\end{document}
