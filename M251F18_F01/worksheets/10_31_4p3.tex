\documentclass[12pt]{article}
\usepackage[top=0.9in, bottom=0.9in, left=0.9in, right=1.1in]{geometry}

\usepackage{graphicx,color,enumitem}
\usepackage{amsmath,amsthm,amsbsy}
\usepackage{palatino}

%% Setup aproblem environment, 
%% aproblem items
%% subproblems environment
%% subproblem items
\makeatletter
\newcounter{probcount}
\newcounter{subprobcount}
\newlength\probsep
\newlength\pshrinking
\newif\iffirstprob
\newenvironment{aproblems}%
  {\ifhmode\unskip\par\fi\setcounter{probcount}{0}\probsep\parskip
  \sbox\@tempboxa{\textbf{9.}}\pshrinking\wd\@tempboxa\advance\pshrinking\labelsep
  \let\hproblem\aproblem
  \advance\linewidth -\pshrinking
  \advance\@totalleftmargin\pshrinking
  \advance\leftskip\pshrinking}%
  {\ifhmode\unskip \par\fi\advance\leftskip-\pshrinking}%

\newcommand{\aproblem}{%
  \setcounter{subprobcount}{0}%
  \stepcounter{probcount}%
  \def\@currentlabel{\arabic{probcount}}%
  \ifhmode
    \unskip \par
  \fi
%  \addpenalty{-4000}%
  \iffirstprob\else\addvspace\probsep\fi
  \firstprobfalse
  \hskip -\labelwidth\hskip -\labelsep 
  \hbox to\labelwidth{\hss\textbf{\arabic{probcount}.}}\hskip\labelsep
}%

\newcommand{\subprob}{\item\def\@currentlabel{\arabic{probcount}\alph{\thelistlabel}}}
\newcommand{\skipproblem}{\stepcounter{probcount}}


%% The following commands put defined left and right headers on the top, and a page number
%% on the bottom of all pages beyond page 1
\usepackage{fancyhdr}
\pagestyle{fancy}
\fancyfoot[C]{\ifnum \value{page} > 1\relax\thepage\fi}
\fancyhead[L]{\ifx\@doclabel\@empty\else\@doclabel\fi}
\fancyhead[R]{\ifx\@docdate\@empty\else\@docdate\fi}
\headheight 15pt
\def\doclabel#1{\gdef\@doclabel{#1}}
\def\docdate#1{\gdef\@docdate{#1}}
\makeatother

%% General formatting parameters
\parindent 0pt
\parskip 6pt plus 1pt


\doclabel{Math F251: Section 4.3 Worksheet}
\docdate{Wednesday 31 October 2018}


\begin{document}
\renewcommand{\d}{\displaystyle}

\begin{aproblems}
% 4.3 #43
\aproblem
    $$F(x) = x \sqrt{6-x}$$
%F' = \sqrt{6-x} + x (1/2) (6-x)^{-1/2} (-1) = \sqrt(6-x) (1 - \frac{x}{2(6-x)})
\renewcommand{\labelenumi}{(\alph{enumi})}
\begin{enumerate}
\item What is the domain of $F(x)$?
\item Find the intervals of increase or decrease.
\item Find the intervals of concavity and the inflection points.
\item Sketch the graph.
\end{enumerate}
\vfill

\aproblem  Compute the following limits; you may use L'Hopital's rule:
    $$\lim_{x\to -\infty} \frac{e^x}{1-e^x} = \hspace{5.0in}$$

\bigskip
    $$\lim_{x\to +\infty} \frac{e^x}{1-e^x} = \hspace{5.0in}$$

\bigskip
(Can you compute the second limit without L'Hopital's rule?  How?)

\newpage
\thispagestyle{plain}
% 4.3 #52
\aproblem
    $$g(x) = \frac{e^x}{1-e^x}$$
\renewcommand{\labelenumi}{(\alph{enumi})}
\begin{enumerate}
\item What is the domain of $g(x)$?
\item Find the horizontal and vertical asymptotes.
\item Find the intervals of increase or decrease.
\item Find the intervals of concavity and the inflection points.
\item Sketch the graph.
\end{enumerate}
\vfill

\end{aproblems}

\end{document}
