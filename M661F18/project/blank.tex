\documentclass[12pt]{amsart}

\usepackage[letterpaper, margin=1in]{geometry}
\usepackage{amssymb}
\usepackage[final]{graphicx}


\begin{document}

\title{Your Title Here}

\author{Your Name}

\date{\today}

\maketitle

\section{Introduction}  PUT CONTENT HERE; PERHAPS CITE SOMETHING \cite{einstein}

%here is how to put in a figure:
%\begin{figure}
%\includegraphics[width=0.5\textwidth]{foo.png}  % put image "foo.png" in current directory
%\caption{CAPTION TEXT HERE}
%\end{figure}

\section{Algorithms [or Examples]}  MORE CONTENT

\section{Examples [or Algorithms]}  CONTENT

\section{Implementation}  CONTENT

\bigskip
\hrule
\begin{verbatim}
% MYCODE  This is my matlab implementation
x = 1:10;
y = randn(size(x));
plot(x,y)
z = 2+2
\end{verbatim}
\hrule
\bigskip

MORE CONTENT

\bigskip
\hrule
\begin{verbatim}
>> mycode             % here I am running the code
z = 4
\end{verbatim}
\hrule

\section{Results}  CONTENT

\section{Analysis}  CONTENT; CITE SOMETHING? \cite{grivanashsofer}

\section{Conclusion}  CONTENT

\begin{thebibliography}{2}  % "2" because there are two references
\bibitem{einstein} 
A.~Einstein (1905). 
\textit{Zur Elektrodynamik bewegter K{\"o}rper},
Annalen der Physik, 322 (10), 891--921.

\bibitem{grivanashsofer}
I.~Griva, S.~Nash, \& A.~Sofer (2009).
\textit{Linear and Nonlinear Optimization},
2nd ed., SIAM Press.
\end{thebibliography}

\end{document}
